\documentclass[11pt]{article}
\usepackage{graphicx}  % Include figure files
\usepackage{amsmath}  % Include figure files
\begin{document}
\title{SBS-offline for SBSGenericDetector  and BBShower}
\maketitle

\section{Introduction}

\subsection{SBSGenericDetector}
SBSGenericDetector is a THaNonTrackingDetector.
For the SBSGenericDetector , a detector can be classified as using ADC and/or TDC information.
The ADC can be different modes: kNone ( no ADC ), kADCSimple ( like a 792 or fastbus), kADC ( F250 in mode 7)
and kWaveform (F250 in mode 1). ADC data from a F250 in mode 7 can have up three pulses recorded
per channel per event. ADC data from a F250 in mode 1 is a vector of 4ns samples. The size of the vector is fixed in the configuration of the FADC readout list.  The size of the vector is determined by
the FADC250 module decoder. 
The TDC can be in modes: kNone ( no TDC for detector), kTDCSimple ( only leading edge), kTDC ( leading and trailing edge). 
The constructor sets variables to default values which are given in Table~\ref{tab:con}. Some variables can be set by functions that can be called from your replay script. 
You can also set this in the constructor of the detector that inherits from SBSGenericDetector. 
If one sets SBSModeADC::Mode not equal to kADC then SetDisableRefADC set to false.
\begin{table}
	\begin{center}
\begin{tabular}{|c|c|c|}
\hline 
\rule[-1ex]{0pt}{2.5ex} Variable & Default  & Comment \\ 
\hline 
\rule[-1ex]{0pt}{2.5ex} fNrows  &  0 &  \\ 
\hline 
\rule[-1ex]{0pt}{2.5ex} fNcolsMax  & 0 &  \\ 
\hline 
\rule[-1ex]{0pt}{2.5ex} fNlayers & 0 &  \\ 
\hline 
\rule[-1ex]{0pt}{2.5ex} fModeADC & kADCSimple  & SetModeADC(SBSModeADC::Mode mode) \\ 
\hline 
\rule[-1ex]{0pt}{2.5ex} fModeTDC  & kNone  & SetModeTDC(SBSModeTDC::Mode mode) \\ 
\hline 
\rule[-1ex]{0pt}{2.5ex} fDisableRefADC  & true  & SetDisableRefADC(Bool\_t b) \\ 
\hline 
\rule[-1ex]{0pt}{2.5ex} fDisableRefTDC  & true &  SetDisableRefTDC(Bool\_t b)\\ 
\hline 
\rule[-1ex]{0pt}{2.5ex} fConst  & 0 &  \\ 
\hline 
\rule[-1ex]{0pt}{2.5ex} fSlope  &0  &  \\ 
\hline 
\rule[-1ex]{0pt}{2.5ex} fAccCharge  & 0 &  \\ 
\hline 
\rule[-1ex]{0pt}{2.5ex} fStoreRawHits  & false & SetStoreRawHits(Bool\_t var) \\ 
\hline 
\rule[-1ex]{0pt}{2.5ex} fStoreEmptyElements  & true &SetStoreEmptyElements(Bool\_t b)  \\ 
\hline 
\end{tabular} 
\caption{Table of variables in the constructor}\label{tab:con}
	\end{center}
\end{table}

\subsection{SBSCalorimeter and SBSBBShower}
SBSCalorimeter contructor inherits from the SBSGenericDetector and leaves the SBSModeADC::Mode as kADCSimple
and SBSModeTDC::Mode as  kNone.
SBSBBShower constructor inherits from the SBSCalorimeter and sets  SetModeADC(SBSModeADC::kWaveform) and
SetModeTDC(SBSModeTDC::kNone).
 



\section{SBSGenericDetector ReadDatabase}
A detector is defined as a grouping of elements which are arranged in rows, cols and layers.
The element is defined by the SBSElement class which stores information about the element
such as its geometry and the TDC and/or ADC data. The TDC and ADC data is stored in structures
and classes defined in the SBSData class.
There are two numbering schemes for vectors of the elements.
Either by element number and like a 3-d array with row, col and layer.
ncols is a parameter that needs to be set. nrows is optional and the default is 1. ncols is an array. 
If the size of ncols array is 1, then all the nrows are given the same number of columns equal to ncols[0].
If the ncols array is nrow size , then each row has a different number of columns.
The nlayers is optional and the default is 1. 

Need to set the vector parameters xyz ( the x,y and z starting position for the
geometry) and dxyz ( the dx, dy and dz spacing between elements).

The parameter start\_chanmap is optional and is not used in the code. The parameter
row\_offset\_pattern is optional and is not used in the code. Maybe these can be eliminated.






The detmap gives list of the crate, slot, start channel , end channel and reference time index
for each module that is used by the detector.
One can also add the module number in the detmap, if one sets 
the parameter model\_in\_detmap = 1. By default model\_in\_detmap = 0 and setting the
detector map is probably only necessary for when using the VETROC.
When one wants to include the model in the detector map, then the detmap format changes
and should be crate, slot, start channel , end channel, model number and reference time index
for each module that is used by the detector.
The reference time index is an integer that indicates which reference time indicated in the chanmap to use
for that module.
The detmap vector is an input to the FillDetMap function of the THaDetMap which produces the object fDetMap.



The chanmap is vector of the element index for the crate, slot and channel. The chanmap
order must follow the ordering of the crate and slots in the detmap.
It is an optional parameter. Without the chanmap defined, the default is to assign the
elements to each channel in the detmap in order.
When creating the chanmap, the element index starts at zero 
and the maximum is the total number of elements ( nrow x ncol x nlayer)
minus one. The chanmap format is to match the number of channels in the detmap and
in the same order of the detmap. If a channel is not used by the detector then put a -1
in the chanmap. If the channel is a reference time, then put a -1000 in the chanmap.

The code loops through the modules in the fDetMap to create a fChanMap and fRefChanMap based on the chanmap.
 fChanMap and fRefChanMap are use in Decode method to determine which channel info is loaded into which element.
 This determines the number of elements, fNelem, and the number of reference time elements for the TDC,
 NRefADCElem and ADC, NRefADCElem.
The vector size of fRefChanMap is equal to the number of -1000 channels that are found and the index of the
fRefChanMap vector is the order that it finds the -1000. So in the detmap , the reference time index refers
to the order of the -1000 in the chanmap and goes from 0 to the number of chanmap =-1000 minus one.
As an example for the FADC, there can be one reference time in one module ( reference time index = 0)
which is used by all FADC modules ( so detmap would have reference time index = 0 for each FADC).
Similarly for the TDC, there can be one reference time in one module to be used for all modules
or a reference time for each module. For the case of FADC and TDC modules, if the TDC modules
are first in the detmap and have one module with  chanmap = -1000 then the TDC detmap would 
use reference time index = 0 and if the FADC modules with  chanmap = -1000 then the ADC detmap
would use reference time index = 1. 
 
The code then reads in parameters used to process the TDC or ADC raw data into calibrated data
for each element. All these parameters are optional with default values in the code.
 For the TDC, the parameters are given in Table~\ref{tab:tdcparam}. The parameter
 vector can have all elements set individually or have one value which is used for all elements.
 The same parameters can be read-in for the TDC reference time but the "tdc" is replaced by "reftdc".
\begin{table}[h]
\begin{center}
	\begin{tabular}{|c|c|c|}
		\hline 
		Parameter name	& Description &  Global variable\\ 
		\hline
		tdc.offset &  offset (default=0)  & tdc\_offset (unit channels)\\
		\hline
       tdc.cal &  calibration (default=0) & tdc\_cal (ns/channel) \\
		\hline
       tdc.GoodTimeCut &  Expected Good Time peak (ns) (default=0) & tdc\_GoodTimeCut \\
\hline
	\end{tabular} 
\caption{Table of TDC parameters}\label{tab:tdcparam}
\end{center}
\end{table}

For the ADC, the parameters for the kADCSimple are given in Table~\ref{tab:adcsimpleparam}.
\begin{table}[h]
\begin{center}
	\begin{tabular}{|c|c|c|}
		\hline 
		Parameter name	& Description &  Global variable\\ 
		\hline 
		adc.pedestal	&  Pedestals ( not used for waveform) & adc\_ped \\ 
		\hline 
		adc.gain	&  gain to convert to GeV & adc\_gain\\ 
		\hline
	\end{tabular} 
\caption{Table of kADCSimple parameters}\label{tab:adcsimpleparam}
\end{center}
\end{table}
For processing the kWaveform ADC data to extract the pulse information , there are the following parameters
given in Table~\ref{tab:adcwaveparam} are used in addition to the kADCSimple parameters. The parameter vectors can have all elements set individually or have one value which is used for all elements.  
The same parameters can be read-in for the ADC reference time but the "adc" is replaced by "refadc".
\begin{table}[h]
	\begin{center}
		\begin{tabular}{|c|c|c|}
		\hline 
        adc.conv	&  conversion factors  (waveform is mV/chan) & adc\_conv\\ 
		\hline 
adc.thres	&  threshold  (mV) & adc\_thres\\ 
\hline 
adc.FixThresBin	&  fixed threshold crossing bin  & adc\_FixThresBin\\ 
\hline 
adc.NSB	&  \# of integration samples before threshold crossing  & adc\_NSB\\ 
\hline 
adc.NSA	&  \# of integration samples after threshold crossing  & adc\_NSA\\ 
\hline 
adc.NPedBin	&  \# of samples for pedestal determination  & adc\_NPedBin\\ 
\hline 
	\end{tabular} 
\caption{Table of ADC parameters}\label{tab:adcwaveparam}
\end{center}
\end{table}



The code creates a vector of elements called fElements based on the SBSElement class.
The element number is incremented
by looping through  number of columns inside the looping through rows.
The element is created and loads the x,y,z locations along with the row, column, layer and element number.
The element can be have ADC and/or TDC information depending on how the GenericDetector is defined.
If the GenericDetector has TDC info, then the element is set to have an SBSData::TDC data function and
 the offset, calibration and GoodTimeCut are loaded.
If the GenericDetector has ADC info, then it depends on the ADC mode.
If the ADC mode is kWaveform, the element is set to have an SBSData::Waveform data function and
the pedestal ( which is not used), gain and conversion factor are loaded. For the kWaveform
there is an additional call to load the parameter used to determine ADC signal in the waveform.
If the ADC mode is not kWaveform , the element is set to have an SBSData::ADC and the
pedestal and gain are loaded.
   

The code creates a vector of reference time elements called fRefElements bases on the SBSElement class.
It follows the same logic for creating the reference time elements as the detector elements, 
except that the reference time element will either be a TDC or an ADC reference time element.
When the reference time element is created the row number and element number are both
 the reference time index.
 
 



\subsection{BBShower}
The BB Preshower is 26 rows of 2 columns in one layer.  The BB SHower is 27 rows of 7 columns in one 
layer.
 For the BB Preshower, the left side was put in slot 3 and slot 4 (chan 9)  and the right side slot 3 ( chan 10)
 to slot 6 ( chan 3). The left side is column 0 , so all the even element indices. The right side is 
 column 1, so all the odd element indices. 
 For BB Shower, the signals are in slot 6 (chan 4) to slot 20 (chan 0). As an example, row 0 and col 0-6
 is in slot 6 and channels 4 to 10.


\section{SBSGenericDetector Decode}

First the ClearEvent method is called. This clears the ADC and/or  TDC data  in all the elements.

If the reference times are in the channel map, then the code looks for the reference time in the data.
The code loops the channels in each module. If there is event data for the reference time channel,
then the data is processed for the FRefElement with DecodeADC or DecodeTDC depending if the module is an ADC or TDC.
Only expect ADC reference times for the FADCs. For ADC waveform, the time for the pulse in 
the reference time is determined the same way as the time a regular detector element.
The TDC reference times are processed the same as a regular detector element.
The total number of  ADC and TDC reference channels hit in the Decode method is fNRefhits.

The code loops through the  channels in each module 
If there is event data for the detector channel,
then the data is processed for that fElement  with DecodeADC or DecodeTDC 
depending if the module is an ADC or TDC.
The total number of  ADC and TDC channels hit in the Decode method  is fNhits. The ADC and TDC channels
are not counted separately. {\bf Look into adding counter for the ADC and TDc separately and put in the tree.}

\subsection{SBSGenericDetector DecodeTDC}
The method assumes that the reference time elements will be processed first in the Decode method.

If a detector time is being processed, then the method first determines the good reference
time to use for the detector element based on the module's reference time index. 
It loops through the reference time TDC hits and selects the reference time hit that has
the smallest abs(Tdc -GoodTimeCut). The GoodTimeCut is in ns. This good reference time is the raw time
in units of channels and is locally stored in the variable reftime. If a reference time is being processed,
then reftime is set to zero.

For either a reference or detector time, the method loops through the hits for that element 
and first determines whether the hit is leading (LE)  or trailing edge (TE).
If the first hit is a TE, then it skips that hit. If the last hit is a LE, then that it skips that hit.
If it has passed the first two tests, then the hit is processed using the SBSData method TDC::Process.
TDC::Process has the arguments of the element ID, raw time value and edge value (LE = 0, TE = 1).
The raw time value that is loaded is the raw time value from the channel minus reftime (local variable determined earlier in DecodeTDC).

TDC::Process fills the fTDC which is a SBSData::TDCData container for that detector or reference time element. 
TDCData holds the offset, calibration, GoodTimeCut and the good\_hit. The  offset, calibration, GoodTimeCut
are set in SBSGenericDetector::ReadDatabase. good\_hit is an integer variable
which holds the index of the good TDC hit that is determined by the SBSGenericDetector::FindGoodHit method 
that is described later. TDCData also contains a vector, "hits"  of the SBSData::TDCHit structure. 
The  SBSData::TDCHit structure contains the "le" SingleData structure, "te" SingleData structure,
"ToT" SingleData structure and "elem ID" ,the integer of the element ID. The SBSData::SingleData structure contains
the "raw" and "cal" (the raw and calibrated values). For a TDC, the raw value is reference time subtracted and in units of channels and the calibrated value is the (raw value - offset)*cal.

TDC::Process has a counter array fEdgeIdx[edge] with edge = 0 (LE) or 1 (TE).
If the TDC detects LE and TE, then the goal is to first find a LE hit and match it with a TE hit which usually
follows the LE.
If the counter fEdgeIdx[edge] is larger or equal to the size of the TDCHit hits vector for that element,
then TDC::Process increments the size of the hits vector and the fEdgeIdx[edge] counter is incremented.
The element ID is loaded into the hits vector.
 If the edge is LE then the "le" structure is filled. If the edge is TE then the "te" structure is filled.
 If the fEdgeIdx[0] equals fEdgeIdx[1] then the "ToT" structure is filled with the "te" - "le".
  If the fEdgeIdx[1] is larger than fEdgeIdx[0], then fEdgeIdx[0] is set equal to fEdgeIdx[1]. This
  is done to insure that if the next hit is a LE then the size of the hits vector will be increased
  whether the next hit is a LE or TE. 
  The TDCHit hits vector is ideally assuming that it will have an LE and TE (or just LE if the TDC module
  was programmed that way). If there is only a LE or TE filled bu a hit, then the missing information will have
  a value of zero. {\bf SHould the deafult value be kBig?}
  
  After looping through all the hits for that element, the element will have an TDC data structure
  that contains the TDCHit hits vector. In the SBSGenericDetector::CoarseProcess, there is a
  call to   SBSGenericDetector::FindGoodHit for each element that has TDC data and the hit which
  is the "good" hit is chosen.
    




\subsection{SBSGenericDetector DecodeADC}
If ADC mode is kADCSimple, the SBSData::ADC Process method is called with a single argument.
It  fills the fADC which is a SBSData::ADCData container for that detector or reference time element. 
The SBSData ADCData structure holds the pedestal, calibration, time calibration ( for FADC) 
and the amplitude calibration. In addition, ADCData contains vector of the PulseADCData structure, hits, and
 the good\_hit which is the index number of the good
hit which are needed for multihit FADC.  For a kADCSimple mode , only the vector will just
be the single hit. The PulseADCData is a structure containing the integral, time and amplitude
which are all SingleData structures having the "raw" and "val" ( calibrated value).
For the kADCSimple mode, the ADC::Process with single argument fills the SingleData integral with the raw data 
and the (raw-ped)*cal with the time and amplitude set to zero and sets fHasData to true.

{\bf The ADC mode kADC which is for mode = 7 in the FADC250 needs to be worked on}


When ADC is in waveform mode, a sample vector is filled with ADC value for
each 4ns bin. The SBSData::Waveform::Process is called.
A vector with the raw ADC converted to mV is filled. The pedestal (in mV) is determined by averaging the first NPedBin ADC values. A vector with the pedestal subtracted ADC is filled.
The threshold crossing bin is found by looping through the sample raw ADC vector and finding the first bin which has a raw ADC value larger than the pedestal plus the threshold.
If a threshold crossing bin, TC, is found, then the ADC samples are integrated from TC-NSB to TC+NSA-1.
If no  threshold crossing bin, TC, is found, then the ADC samples are integrated from FixThresBin-NSB to FixThresBin+NSA-1.
The ADC integration is converted to pC using the 4ns bin width and 50 ohm resistance.

If a threshold crossing bin is found, then the peak amplitude is found within the integration region. The peak, Vpeak,  is the ADC value in the sample which within the integration window (starting from the threshold crossing) the following bin has a smaller ADC value than previous sample.
To calculate the pulse time, first  $V-{Mid} = (V_{peak} + \mbox{pedestal})/2$ is calculated. Then loops through the
integration region and finds the sample, i, with which has value greater than Vmid and the next sample, i+1, is
less than Vmid. The time is $4\mbox{ns}*\left[(i+(V_{Mid}-V_i)/(V_{i+1}-V_i)\right]$.
If no threshold crossing bin is found, then the time and amplitude are set to zero.
For now the code only find the first pulse that is above threshold.






\subsection{SBSGenericDetector::CoarseProcess}
The CoarseProcess method separately loops through the reference time elements and the detector elements.
The goal is to fill the fRefGood and the fGood SBSGenericOutputData structures. The
SBSGenericOutputData is a structure the contains vectors of the row,col,layer, element ID.
SBSGenericOutputData contains vectors of the ADC integral, pedestal subtracted ADC integral, 
the calibrated   pedestal subtracted ADC integral, FADC pulse amplitude, FADC pedestal subtracted pulse amplitude
and the FADC pulse time. SBSGenericOutputData contains vectors of the TDC LE time, TE time and Tot ( TE-LE) time.
 SBSGenericOutputData contains vectors of  number of samples, index for the start of 
 samples and the samples of the full FADC waveform.
 
 The method first loops through the reference time elements. The fRefGood structure 
 is filled with row, column, layer and element ID. If the reference time element is
 a TDC and has TDC data, then the TDCHit structure for the good reference time hit is retrieved
 and the fRefGood TDC structure is filled.
 By default SBSGenericDetector sets fStoreEmptyElements to true, but the detector class that inherits 
 could set  fStoreEmptyElements to false or this can be set in the replay script.
  If  fStoreEmptyElements is true then the TDC structure is filled with kBig.
 
If the element has TDC data then it is possible have the code store the TDC hit information for
all hits in the reference time elements in the fRefRaw SBSGenericOutputData structure.
 By default SBSGenericDetector sets fStoreRawHits to false but the detector class that inherits 
 could set  fStorerawHits to true or this can be set in the replay script.
 If true, then the fRefRaw struture is filled with the element ID and the calibrated LE, TE and Tot values
 for all hits in an element for all elements in the loop that have TDC data.
 
 {\bf Need to add code for processing the ADC reference times}
 
 The method then loops through all the detector elements. {\bf It calls SBSElement:CoarseProcess which fills
 the energy for the element. This should be eliminated?}.
The SBSGenericDetector::FindGoodHit method is called for the element. If the element has TDC data
then the "good" TDC hit is determined by finding the TDC hit which has the smallest abs(Tdc -GoodTimeCut). The GoodTimeCut is in ns. {\bf If the element has ADC data, then first hit is used. Presently no multihit with
the waveform} The detector class that inherit from GenericDetector could implement their own FindGoodHit method.


After the good hit is determined, if the element has TDC data, then fGood is filled with TDC information. If  fStoreEmptyElements is true then the TDC information is filled with kBig. If fStoreRawHits is true, then all the hits in the element is
loaded into the fRaw SBSGenericOutputData structure.

Next, if the element has ADC data (fHasData is true), then the "good" ADC 
hit is used to fill the fGood with ADC information.
If the ADCmode is kADCSimple, then the fGood pedestal, raw integral ("a"), raw -pedestal ("a\_p")
and the calibrated value ("a\_c") are loaded. For kADC mode, the amplitude ("a\_amp"),
amplitude - pedestal ("a\_amp\_p") and pulse time ("a\_time") are loaded. {\bf need to work on this but at this point noone uses this adc mode}. If fStoreRawHits is true ( by default it is false) then the calibrated integral,
calibrated amplited and pulse time are loaded into fRaw structure. {\bf Need to add the ADC element ID, since it will be different than the TDC element ID}
 If  fStoreEmptyElements is true ( it is set true by default) then the ADC information is filled with zero.
 If the ADC mode is Waveform, then the  fGood pedestal, raw integral ("a"), raw -pedestal ("a\_p"), 
 the calibrated value ("a\_c"), the amplitude ("a\_amp"),
 amplitude - pedestal ("a\_amp\_p") and pulse time ("a\_time") are loaded. In addition, the
 index of the sample data in the "samps" vector ("sidx"), the number of samples per element ("nsamps") and the
 calibrated sample data ("samps") are loaded into a vector for all elements.
 {\bf Presently, there is no check on a "good hit", since there is only one hit determined per waveform.}
 
 In general, a detector class that inherits from the GenericDetector should first call GenericDetector::CoarseProcess and the used the fGood structure to process the data for 
 further analysis.
 

 
 

\subsection{SBSGenericDetector Tree variables}

For the tree variables, most are arrays of variable size depending on the number of hits. 
The total number of  ADC and TDC channels hit in the Decode method is fNhits. The ADC and TDC channels
are not counted separately.
\begin{center}
	\begin{tabular}{|c|c|c|}
		\hline 
		Tree name	& Description &  Global variable\\ 
		\hline 
		nhits	& Number of total hits & fNHits  \\ 
		\hline 
		ngoodhits	& Number of total good hits & fNGoodhits \\ 
		\hline 
\end{tabular} 
\end{center}

If fStorerawHits is set true, then the ADC/TDC raw info is stored.

\begin{center}
\begin{tabular}{|c|c|c|}
	\hline 
	Tree name	& Description &  Global variable\\ 
	\hline 
	hits.t 	&  all hit's Calibrated Leading edge TDC  &fRaw.t  \\ 
	\hline 
	hits.t\_te 	& all hit's  Calibrated Trailing edge TDC &fRaw.t\_te  \\ 
	\hline 
	hits.t\_tot	&  all hit's TDC Time Over Threshold   &fRaw.t\_ToT  \\ 
	\hline 
\end{tabular} 
\end{center}
\begin{center}
	\begin{tabular}{|c|c|c|}
		\hline 
		Tree name	& Description &  Global variable\\ 
		\hline 
		hits.a 	&  all ADC integral of the all hits  (units pC)  &fRaw.a  \\ 
		\hline 
		hits.a\_amp 	&  all ADC peaks of the all hits  (units mV)  &fRaw.a\_amp  \\ 
		\hline 
		hits.a\_time	&  all ADC time of the all hits (units mV)  &fRaw.a\_time  \\ 
		\hline 
	\end{tabular} 
\end{center}

The good hit information is the following variables.

\begin{center}
\begin{tabular}{|c|c|c|}
	\hline 
Tree name	& Description &  Global variable\\ 
	\hline 
nhits	& Number of total hits & fNHits  \\ 
	\hline 
ngoodhits	& Number of total good hits & fNGoodhits \\ 
	\hline 
row 	& Row of the detector element  & fGood.row  \\ 
	\hline 
col 	& Column of the detector element  & fGood.col \\
	\hline 
layer	& Layer of the detector element  & fGood.layer \\ 
	\hline 
ped	& Pedestal of the good hit (units mV) & fGood.ped \\ 
	\hline 
\end{tabular} 
\end{center}

\begin{center}
\begin{tabular}{|c|c|c|}
	\hline 
Tree name	& Description &  Global variable\\ 
	\hline 
a	& ADC integral of the good hit (units pC)  &fGood.a  \\ 
	\hline 
a\_p	& Pedestal subtracted ADC integral of the good hit  (units pC) & fGood.a\_p \\ 
	\hline 
a\_c	& Energy Calibrated Pedestal subtracted ADC integral of the good hit (units GeV) & fGood.a\_c \\ 
\end{tabular} 
\end{center}

\begin{center}
\begin{tabular}{|c|c|c|}
	\hline 
Tree name	& Description &  Global variable\\ 
	\hline 
a\_amp	& ADC peak of the good hit(units mV)  &fGood.a\_amp  \\ 
\hline 
a\_amp\_p	& Pedestal subtracted ADC peak of the good hit (units mV) & fGood.a\_amp\_p \\ 
\hline 
a\_time	& ADC time of the good hit  (units ns)  &fGood.a\_time  \\ 
\hline 
a\_amp	& ADC peak of the good hit (units mV)  &fGood.a\_amp  \\ 
\hline 
\end{tabular} 
\end{center}


\begin{center}
\begin{tabular}{|c|c|c|}
	\hline 
	Tree name	& Description &  Global variable\\ 
	\hline 
	tdc 	& Calibrated Good hit Leading edge TDC  &fGood.t  \\ 
\hline 
	tdc\_te 	& Calibrated  Good hit Trailing Edge TDC  &fGood.t\_te  \\ 
\hline 
	tdc\_tot 	& Calibrated  Good hit Time over Threshold  &fGood.t\_Tot  \\ 
\hline 
\end{tabular} 
\end{center}


\begin{center}
	\begin{tabular}{|c|c|c|}
	\hline 
	Tree name	& Description &  Global variable\\ 
	\hline 
	samp\_idx 	& Index in the samples vector for given row-col element &fGood.sidx  \\ 
	\hline 
	samp\_nsamps	& Number of samples for a given row-col element &fGood.nsamps  \\ 
\hline 
	samp 	& Calibrated Pedestal subtracted ADC samples (units mV) in 4ns bins &fGood.samps  \\ 
\hline 
\end{tabular} 
\end{center}

%\section{ Todo list}
%\begin{itemize}5
%	\item In SBS5GenericDetector::Decode have variables to count number of ADC and TDC hits separately. If ADC %has multiple5 pulses  TDC will have multiple hits. 
%	\item In SBS5Data::Process 
%\end{itemize}



\section{Short description of FADC250}
The FADC250 is a fast data acquisition electronics module which is able to sample a signal pulse at 250MHz (i.e. 4ns time sample).
It has two main data acquisition modes:
%
\begin{itemize}
	\item{Mode 1: all samples within a defined time window are recorded; this allows to perform the waveform reconstruction offline.}
	\item{Mode 7: if the recorded pulse passes a defined threshold, the following quantities are recorded: the pedestal, the pulse time, the pulse amplitude, and the pulse integral over a defined time window; Up to 3 pulses can be recorded the trigger window}
\end{itemize}
%
For the SBS hadron calorimeter (HCal) and SBS BBShower/PreSHower, the FADC will be used with acquisition mode 1.

Details on the algorithm of the FADC is given in the manual. Here is a short description.
In mode 7, a threshold is defined in the ROL FADC  configuration file for each channel in the FADC. When a pulse has one sample that passes the threshold, then that sample is marked
as the threshold crossing sample, TC.  The pulse is integrated from NSB samples before the TC and NSA samples after the TC. NSB and NSA are programmed in the ROL FADC configuration file.
The average of first four samples in the trigger window are used to determine the pedestal or VMIN.
The peak amplitude (VPEAK) is determined by finding a sample beyond TC for which the sample value first decreases.
The fine time, TF,  is determined in bins of 62.5ps. First the time bin is roughly determined by the time bin of the half amplitude (VMID = (VPEAK + VMIN) / 2)).
Then the fine time is determined as TF = 64*(VMID – V(N1)) / (V(N1+1) – V(N1)). The sample number N1 is found on the leading edge of the pulse that satisfies
V(N1) $\leq$ VMID $<$ V(N1+1).

\end{document}


