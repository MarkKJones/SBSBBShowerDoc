\documentclass[11pt]{article}
\usepackage{graphicx}  % Include figure files
\usepackage{amsmath}  % Include figure files
\usepackage{xcolor}
\usepackage{tabularx}
\begin{document}
\title{SBSCalorimeter}
\maketitle

\section{Introduction}
SBSCalorimeter contructor inherits from the SBSGenericDetector and leaves the SBSModeADC::Mode as kADCSimple
and SBSModeTDC::Mode as  kNone.
There is no separate Decode or CoarseProcess  method which are in  SBSHcal and SBSTotalShower.
There is a ClearEvent method.
The MakeGoodBlocks method for creating the fGoodBlock output data structure and the internal fBlockSet.
The FindClusters method.

\section{ReadDatabase}
First it calls SBSGenericDetector::ReadDatabase.
All additional parameters are optional.
\begin{table}[h]
\begin{center}
\begin{tabularx}{\textwidth}{|c|X|c|}
		\hline 
Parameter name	& Description &  Global variable\\ 
\hline
emin & minimum energy (GeV) to include in cluster & fEmin \\
\hline
cluster\_dim &  vector of x and ymax cluster size &  cluster\_dim \\
\hline
nmax\_cluster & Max number of clusters to store & fMaxNclus \\
\hline 
const & constant for gain correction & fConst \\
\hline 
slope & slope for gain correction & fSlope \\
\hline 
acc\_charge & Accumulated charge & fAccCharge \\
\hline
\end{tabularx} 
\end{center}
\end{table}

\section{MakeGoodBlocks method}
Loops through all fElements to fill the fGoodBlocks which is an
output SBSCalBlocks object which is a structure of vectors. If element is expecting TDC and ADC,
then fills the fGoodBlock structure if both TDC and ADC of good\_hit index not equal to -1.
If element is only expecting ADC, then fills the fGoodBlock structure if ADC of good\_hit index not equal to -1.
After filling the fGoodBlocks stucture, then fills fBlockSet vector of SBSBlockSetList structures which
is used in FindClusters method. The fBlockSet is filled if the fGoodBlock has an energy greater than fEmin.
The  SBSBlockSetList structure has the row, col, id , x, y, TDCTime, e and ADCTime.
The fBlockSet is sorted by energy.
\begin{table}[h]
	\begin{center}
		\begin{tabularx}{\textwidth}{|c|c|X|}
			\hline 
			Variable	& Description &  Element method\\ 
			\hline 
			row & Row & GetRow() \\
			\hline
			col & Column & GetCol() \\
			\hline	
		  id & Element ID & GetID() \\
		  \hline
		  x & x position (m) & GetX() \\
		  \hline
		  y & y position (m) & GetY() \\
		  \hline
		   TDCTime & LE tdc time (ns) & fTDC[good\_hit].le.val \\
		   \hline
		   e  & Energy (GeV) & fADC[good\_hit].integral.val or GetIntegral().val \\
		   \hline
		   ADCTime & ADC time( ns) & fADC[good\_hit].time.val or GetTime().val \\
		   \hline
\end{tabularx} 
\end{center}
\end{table}

\section{FindClusters}
The method loops through the fBlockSet until each element is assigned to a cluster.
Cluster information is stored in fClusters which is a vector of SBSCalorimeterCluster objects.
The first cluster is started with the element that has the maximum energy and the first
element in fCluster is created and the vector element is removed from fBlockSet.
A while loop is started for adding additional detector elements in fBlockSet to the present fCluster.
Inside the while loops, the code loops through fBlockSet until is finds a neighbor or 
reaches the end of fBlockSet.
If it finds a neighbor, then the neighbor is added to the present fCluster element; 
the vector element is removed from fBlockSet; and it again fBlockSet until is finds a neighbor or 
reaches the end of fBlockSet.
If it reaches the end of  fBlockSet then it stops adding detector elements to the present fCluster.
If there are detector elements remaining in the fBlockSet, then another cluster is created from the
first element in the fBlockSet which has been sorted by energy. 
 This continues until  each element in fBlockSet has been assigned to a cluster.
 
 Presently the neighbor is determined if the block is within given radius of the current cluster x and y
 position. {\color{red} Need to update this and have parameters in ReadDatabase}
 
 \section{SBSCalorimeterCluster}
 SBSCalorimeterCluster is a class for storing calorimeter cluster information which is updated 
 when a new element is added to the cluster.
 The class stores the number of blocks in the cluster, fMult. It stores the total energy, energy weighted x
 and y positions (fE, fX and fY)  
 It stores the energy, row and col of the block with the highest energy.
 It stores a SBSElement vector of the elements in the cluster and a pointer to the element with the maximum energy.
 {\color{red} Need to add the TDC and ADC time information. } 
 
 
\section{FineProcess}
It first calls SBSGenericDetector::FineProcess and then it fills
various cluster structures for the tree output.
The cluster information for tree output is saved in the SBSCalorimeterOutput structure.
The vectors in the structure are given in Table.
The fMainClus is a  SBSCalorimeterOutput structure for the cluster with the maximum energy
os the vectors always have size of 1. There is a flag, fDataOutputLevel, which controls the
additional output. This flag can be set with the function: SetDataOutputLevel(Int\_t).
If fDataOutputLevel $>$ 0, then the SBSCalorimeterOutput structure fMainClusBlk is filled
in which information of each block that is in the main cluster is loaded into the vectors.
If fDataOutputLevel $>$ 1, then  the SBSCalorimeterOutput structure fOutClus is filled with
the cluster information for each of the clusters.

\begin{table}[h]
	\begin{center}
		\begin{tabularx}{\textwidth}{|c|X|c|}
			\hline 
Tree name	& Description &  Global variable\\ 
\hline
e & Energy of cluster &  \\
\hline
e\_c & Corrected energy of cluster & \\
\hline
x & x position (m) of cluster & \\
\hline
y & y position (m) of cluster & \\
\hline
blk\_e & Energy of block with max energy &  \\
\hline
blk\_e\_c & Corrected energy of block with max energy  & \\
\hline
row 	& row of block with max energy & \\	    
\hline
col 	& column of block with max energy & \\	    
\hline
id 	& Element ID  of block with max energy & \\	    
\hline
\end{tabularx} 
\end{center}
\end{table}




\end{document}

